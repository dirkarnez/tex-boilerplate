\documentclass[12pt]{article}

\usepackage{hyperref}
\usepackage{tabularx}
\usepackage{xurl}
\usepackage{mathptmx}
\usepackage{setspace}

\doublespacing

\urlstyle{same}

% % main details
\title{Adonis}
\author{Nicholas Mamo \textsuperscript{1}}

\begin{document}
	\maketitle
	
	\section{Introduction}
		The best way to appreciate a good template is to force yourself to write with a bad template. 
    This is the \textit{Adonis} template.
		Its design stems from a personal experience.
		I was writing a manuscript that I had been planning for months, about a project that I had been developing for years, and on a subject I adored.
		And I dreaded every minute I spent drafting the manuscript.
		It took a while until I realized why: the template felt entirely off-putting.
		
		The template should elevate the writing, not diminish it.
		In academia, however, form normally follows function.
		Sometimes it feels like publishers deliberately diminish the form in a vain attempt to elevate the function: the words, the science, and nothing else.
		Having a good writing environment matters.
		To design the \textit{Adonis} template, I followed three principles:
		
		\begin{itemize}
			\item Simplicity, by which I mean several things.
				  I mean that I wanted the template to be simple for me to develop, lest it turn into an exercise in procrastination.
				  I also mean that it should be easy for you, the writer, to use and adapt.
				  Above all, I mean that it should be easy for your reader to consume.
			
			\item Readability, by which I mean readability throughout the writing process.
				  The template should to make it easy to draft manuscripts, revise and read.
			
			\item Aesthetic, by which I mean elegant.
				  Simple and readable \LaTeX{} templates abound, but when I looked, I found most to favour function over form.
		\end{itemize}
	
		The rest of this guide documents design considerations for the layout, typography and other elements.
	
        \section{Conclusion}
		I designed \textit{Adonis} to be as simple to use as possible.
		The optional commands, for example, mean that you do not have to define everything at once; you can simply start writing.
		To make the template easier to use, \textit{Adonis} also comes with a separate file, \texttt{quickstart.tex}, without text, commented-out commands and space to write.
		
		I hope that you find this template to elevate both form and function, and that it proves it possible for the two to co-exist.
		If you find any issues in \textit{Adonis}, or if you have suggestions to make it better, you can reach out to me at the email on the first page, or by opening an issue on the template's repository~\cite{repository}.
    
	\begin{thebibliography}{5}
		\bibitem{repository}
		Adonis template. Nicholas Mamo (2023). \url{https://github.com/NicholasMamo/adonis-template}
	\end{thebibliography}
	
\end{document}